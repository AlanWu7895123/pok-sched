%
%                               POK header
%
% The following file is a part of the POK project. Any modification should
% be made according to the POK licence. You CANNOT use this file or a part
% of a file for your own project.
%
% For more information on the POK licence, please see our LICENCE FILE
%
% Please follow the coding guidelines described in doc/CODING_GUIDELINES
%
%                                      Copyright (c) 2007-2022 POK team

\chapter{Getting started}

   \section{First experience with POK}
   To build and run your first system with POK, you must have the \ocarina code
   generator\footnote{Available at http://aadl.telecom-paristech.fr} and all the
   software required by POK (a list is available in the chapter
   \ref{chapter-installation}).

   Then, perform the following actions :
   \begin{enumerate}
      \item
         Issue \texttt{make configure} at the top directory of POK. Is something
         is missing, install it !
      \item
         Enter the directory \texttt{examples/partitions-threads} by typing
         this command in a terminal:
         \begin{verbatim}
         cd examples/partitions-threads
         \end{verbatim}
      \item
         Invoke \texttt{make}. It will generate configuration and application
         code with \ocarina.
      \item
         Invoke \texttt{make run} in the \texttt{generated-code} directory. You
         can do that with the following commands
         \begin{verbatim}
         make -C generated-code run
         \end{verbatim}
         Using this command, \textit{qemu} is launched and your system is being
         executed.
   \end{enumerate}

   Now, the next sections will explain how to configure the kernel and the
   partition layer for your own projects.

   \section{Development cycle}
   POK has a dedicated development cycle which avoid the compilation of the
   kernel. The development process automatically compiles the kernel, the
   partitions with the application code and assemble them into a bootable
   binary (you can see the illustration of this development process in figure
   \ref{figure:partitioned-architecture-build}.

   Due to the tedious configuration efforts of each layer, a tool that automatically
   configures the kernel and the partitions from \aadl descriptions is available
   (the \ocarina code generator). You can
   also configure each part by yourself by writing C configuration code.

   \section{Configure POK: the \texttt{conf-env.pl} script}
   \label{pok-configuration}
   POK distribution can be configured so reach different goals. The basic
   configuration is automatically performed. However, in some cases, you want to
   use some additional options.

   At first, the configuration of POK is made with the \texttt{conf-env.pl}
   script, located in the \texttt{misc} directory. So, issue
   \texttt{./misc/conf-env.pl} to use the default configuration. The
   configuration is automatically produced by this script and written in the
   \texttt{misc/mk/config.mk} file.

   Then, the \texttt{conf-env.pl} script can be used with additional switches
   to enable some options of POK. There is a list of these switches:
   \begin{itemize}
      \item
         \texttt{--help} : print help menu
      \item
         \texttt{--with-xcov} : use xcov from the coverage\footnote{see
         \texttt{http://forge.open-do.org/projects/couverture/}} project. With
         this option, when you invoke \texttt{make run} after building a system,
         the emulator will be stopped after 40 seconds of execution and analyses
         the code coverage of the system.
      \item
         \texttt{--with-floppy} : add an additional rule so that you can
         automatically install the POK binary into a bootable floppy. In
         consequence, you can invoke \texttt{make install} in the generated
         directory to  create this floppy disk image.
      \item
         \texttt{--with-instrumentation} : automatically insrument kernel and
         partition code and produce additional output to trace system activity.
         This functionnality produces additional files to trace and analze POK
         behavior with third-party tools such as Cheddar\ref{annex-url}.
   \end{itemize}

   \section{Automatic and manual configuration}
   The automatic code generation finely configure the kernel and enable only
   required functionnalities. If is especially efficient for embedded systems
   when you have to avoid useless features and reduce the memory footprint.
   In addition, it avoids all potential errors introduced by the code produced
   by human developers. The automatic configuration process is detailed in
   chapter \ref{chapter-automatic-configuration}.

   On the other hand, you can also configure the kernel and the partitions by
   yourself. In this case, the configuration will be very difficult since POK
   has many configuration directives. This configuration process is detailed in
   the next section.

   \section{Kernel configuration with ARINC653 XML files}
   You can also configure the kernel with an ARINC653 XML file. The tool is
   available in POK releases in the \texttt{misc/} directory. More information
   can be found in section \ref{section-arinc653-xml-configuration}.

   \section{How to write my manual code ?}
   At this time, if you try to write the configuration code by yourself, you
   have to read the configuration directives of POK. The fact is that you need
   to write the configuration code by yourself and make your own build system
   that supports POK (the automatic configuration process output code and
   automatically create the build system for you).

   In that case, the best is to start from a working example. Try to take the
   generated code from the \texttt{examples} directory. It could be efficient
   since they are many examples that use various services of the runtime.

   Finally, the \pok team plans to release a tool that would help the developper
   in the configuration of the kernel and partitions. Such a tool would be
   graphic (like the well-known \texttt{make menuconfig} of the \linux kernel)
   and would propose to configure kernel and partitions.

\clearpage

   \section{Using Ada for partitions}
   Both C and Ada can be used for partitions. Ada will nevertheless require
   some tuning to run into POK, only a GCC toolchain that handles Ada is
   needed.

   Since POK partitions are loaded by executing their main function, one of
   the Ada packages must export a function as \texttt{main}. Moreover, the
   runtime should be disabled using \texttt{pragma No\_Run\_Time}.

   The following piece of code is an example of how to proceed:

   \begin{lstlisting}[language=Ada]
-- main.ads
pragma No_Run_Time;
with Interfaces.C;

package Main is
   procedure Main;
   pragma Export (C, Main, "main");
end Main;


-- main.adb
package body Main is
   procedure Printf (String : in Interfaces.C.char_array);
   pragma Import (C, Printf, "printf");

   procedure Main is
   begin
      Printf ("Hello world!");
   end Main;
end Main;
   \end{lstlisting}

   An ARINC653 layer is also available in \texttt{libpok/ada/arinc653} and
   should be used the same way as described above.

\clearpage

   \section{Run POK on Leon3}
   To build and run POK on Leon3, you must have the TSIM simulator\footnote{Evaluation version available at ftp://ftp.gaisler.com/gaisler.com/tsim/}.

   Then, perform the following actions :
   \begin{enumerate}
      \item
        Add tsim-leon3 directory to you PATH environment variable.
      \item
         Issue \texttt{make configure} at the top directory of POK.
      \item
         Enter the directory \texttt{examples/partitions-scheduling} by typing
         this command in a terminal:
         \begin{verbatim}
         cd examples/partitions-scheduling
         \end{verbatim}
      \item
         Invoke \texttt{make ARCH=sparc BSP=leon3}. It will generate configuration and application
         code with \ocarina.
      \item
         Invoke \texttt{make ARCH=sparc BSP=leon3 run}.
         Using this command, \textit{TSIM} is launched and your system is being
         executed.
   \end{enumerate}

