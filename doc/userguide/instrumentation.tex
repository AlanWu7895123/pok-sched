%
%                               POK header
%
% The following file is a part of the POK project. Any modification should
% be made according to the POK licence. You CANNOT use this file or a part
% of a file for your own project.
%
% For more information on the POK licence, please see our LICENCE FILE
%
% Please follow the coding guidelines described in doc/CODING_GUIDELINES
%
%                                      Copyright (c) 2007-2022 POK team

\chapter{Instrumentation}
\label{chapter-instrumentation}
You can automatically instrument POK using the \texttt{--with-instrumentation} option when 
you configure the build-system (see section \ref{pok-configuration} for more information.
In consequence, when you use this mode, more output is produced and additional files are automatically created when
the system stops. This section details the files that are automatically produced in this mode and how to use them.

   \section{Instrumentation purpose}
   At this time, the instrumentation functionnality was done to observe scheduling of partitions and tasks
   in the Cheddar scheduling analysis tool. If you want to use this functionnality, you have to install Cheddar (see \ref{annex-url}
   for information about Cheddar).

   \section{Output files}
   When you run your system using the \texttt{make run} command and if the instrumentation configuration flag was set, the following files are
   automatically produced:
   \begin{itemize}
      \item
         \texttt{cheddar-archi.xml}: contains the architecture used with the POK kernel.
      \item
         \texttt{cheddar-events.xml}: contains the scheduling events that are registered during POK execution.
   \end{itemize}

   \section{Use cheddar with produces files}
   Start Cheddar. Then, load the XML file \texttt{cheddar-archi.xml}. You can also use Cheddar with this file as an argument.

   Then, load the \texttt{cheddar-events.xml} file. For that, use the menu \texttt{Tool}/\texttt{Scheduling}/\texttt{Event Table Service}/\texttt{Import}
   and choose the generated \texttt{cheddar-events.xml} file.

   Then, to draw the scheduling diagram, choose the menu option \texttt{Tool}/\texttt{Scheduling}/\texttt{Event Table Service}/\texttt{Draw Time Line}.

   Cheddar will directly draw the scheduling diagram that corresponds to the execution.

