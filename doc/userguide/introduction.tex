%
%                               POK header
%
% The following file is a part of the POK project. Any modification should
% be made according to the POK licence. You CANNOT use this file or a part
% of a file for your own project.
%
% For more information on the POK licence, please see our LICENCE FILE
%
% Please follow the coding guidelines described in doc/CODING_GUIDELINES
%
%                                      Copyright (c) 2007-2022 POK team

\chapter{Introduction}

   \section{What is \pok ?}
   \pok is a kernel dedicated to real-time embedded systems. This kernel works on
   various architectures. The configuration code, the deployment code as well as
   the application (userland) code can be automatically generated, achieving
   zero-coding approach.

   One main goal of \pok is to be compliant with many industrial standards. In
   the embedded domain, many API exist for different application domains
   (avionics, railway, automotive). However, concepts remain the same. \pok 
   proposes a canonical adaptive kernel compliant with several industrial
   standards.

   \section{Purpose of this document}
   This document provides general information about \pok. It also present
   available API, describes them and detail how to configure the kernel.

   \section{Supported platforms}
   At this time, \pok supports the following platforms:
   \begin{itemize}
      \item[$\bullet$]
         x86, emulation with \qemu
      \item[$\bullet$]
         \powerpc
      \item[$\bullet$]
         \leon, a platform for aerospace applications
   \end{itemize}

      \subsection{x86}
      The x86 support is included to rapidly develop applications and 
      test them into an emulator like \qemu or bochs.

      \subsection{PowerPC}
      A \powerpc port is available. The port is available for the following BSP:
      \begin{itemize}
         \item[$\bullet$]
            prep
      \end{itemize}

      \subsection{LEON3}
      A port for the LEON3 architecture (a typical architecture in the aerospace
      domain) is currrently in progress. Please contact the POK team if you are
      interested by this port.


   \section{Supported standards}
   \pok is compliant with the following standards:
   \begin{itemize}
      \item[$\bullet$] ARINC653
      \item[$\bullet$] MILS
   \end{itemize}

   To achieve standard compliance, \pok relies on a minimal API that provides few
   canonical services. These services then interact with the kernel to interact
   with other nodes/processes.

      \subsection{ARINC653 support}
      At this time, \pok is compliant with the ARINC653 standard, meaning that it
      provides partitioning functionnalities. On the userland-side, it provides
      all the C and Ada API of the first part of ARINC653.

      However, \pok does not have an XML parser to automatically create the
      configuration/deployment code from ARINC653 XML files. 
     
      \subsection{MILS}
      MILS stands for Multiple Independent Level of Security. \pok, by defining
      strong partitioning, can be MILS-compliant, depending on its
      configuration. Most of the time, the MILS compliant can be reached with
      a analysis of the system in terms of security. To encrypt data, \pok
      relies on the OpenSSL library, released under a BSD-license (see
      http://www.openssl.org for more information).

   \section{About the \pok team}
   \pok is a research project. It was made to experiment partitioned
   architectures and build safe and secure embedded systems. It was initiated
   during a PhD thesis at TELECOM
   ParisTech\footnote{http://www.telecom-paristech.fr} and
   LIP6\footnote{http://www.lip6.fr} laboratories. The developer leader is
   \textsc{Julien Delange}, who did his PhD on partitioned architectures.

   However, several students from the EPITA school\footnote{http://www.epita.fr}
   joined the project and improve for several purposes (projects, exercises, fun,
   \ldots). In addition, other people contributed to the project for several
   reasons.

   There is a list of the people involved in the project (alphabetical order):
   \begin{itemize}
      \item
         Fabien Chouteau (LEON port)
      \item
         Tristan Gingold (PowerPC port)
      \item
         François Goudal (initial version known as the Gunther project)
      \item
         Laurent Lec (Realtek 8029 driver, kernel IPC and so on)
      \item
         Pierre-Olivier Haye (memory protection)
      \item
         Julian Pidancer (initial work on partition isolation)
   \end{itemize}

   Hope the team will grow up in a near future !

